\chapter{Launching}
La fase di avvio del progetto è quella in cui si recluta il team di sviluppo, si conduce il kick-off meeting, si definiscono le regole operative per il team, si gestiscono le comunicazioni e i cambiamenti di scope. In questa fase, il project manager ha il compito di motivare, coordinare e guidare il team verso il raggiungimento degli obiettivi del progetto.

\section{Scelta del Team di Sviluppo}
Per il progetto in questione, il team di sviluppo è stato scelto tenendo conto delle competenze necessarie per realizzare il sistema CityTwin. Il team è composto da:

\begin{itemize}
    \item Un architetto software, che ha il ruolo di progettare l’architettura del sistema, definire le specifiche tecniche, coordinare e supportare gli sviluppatori, garantire la qualità e la sicurezza del codice.
    \item Un esperto in sicurezza informatica, che ha il ruolo di valutare e mitigare i rischi di attacchi informatici, implementare le misure di protezione dei dati e delle comunicazioni, testare la robustezza e la resilienza del sistema.
    \item Due esperti in IoT e sensoristica, che hanno il ruolo di selezionare e configurare i sensori, integrare i dispositivi con il sistema, gestire i dati raccolti, ottimizzare le prestazioni e il consumo energetico.
    \item Otto sviluppatori generici, che hanno il ruolo di sviluppare le funzionalità del sistema, seguendo le indicazioni dell'architetto software e dell'esperto in sicurezza informatica, utilizzando i linguaggi e gli strumenti più adatti.
\end{itemize}

Il team di sviluppo è già consolidato, in quanto i suoi membri hanno lavorato insieme in precedenti progetti, e hanno quindi una buona conoscenza reciproca, una forte coesione e una elevata produttività.

\section{Kick-Off Meeting}
Il kick-off meeting è il primo incontro formale tra il cliente e il team di sviluppo, che segna l'inizio ufficiale del progetto. Lo scopo del meeting è di allineare le parti sulle finalità, gli obiettivi, i requisiti, i vincoli, i benefici e i rischi del progetto, oltre a stabilire le modalità di collaborazione e comunicazione.

Il kick-off meeting si è tenuto in modalità telematica, tramite una piattaforma di videoconferenza. Il meeting è durato circa due ore, ed è stato strutturato nel seguente modo:

\begin{itemize}
    \item \textbf{Introduzione}: il project manager ha dato il benvenuto ai partecipanti, ha presentato il team di sviluppo, e ha illustrato l'agenda del meeting.
    \item \textbf{Presentazione del progetto}: il project manager ha ripassato gli elementi chiave del progetto, utilizzando il POS e la WBS come supporti visivi. Ha chiesto al cliente di confermare la sua comprensione e il suo accordo.
    \item \textbf{Presentazione del piano di lavoro}: il project manager ha presentato il piano di lavoro del progetto, indicando le fasi, le attività, le scadenze, le risorse, i costi e le responsabilità previste. Ha spiegato il modello di project management life cycle (PMLC) scelto per il progetto, basato su un approccio agile, e ha descritto le modalità di assegnazione e monitoraggio dei task, utilizzando uno strumento di gestione dei progetti online.
    \item \textbf{Presentazione del piano di comunicazione}: il project manager ha presentato il piano di comunicazione del progetto, specificando gli stakeholder coinvolti, le informazioni da comunicare, la frequenza e la modalità della comunicazione, gli strumenti e i canali della comunicazione, e le responsabilità e i ruoli nella comunicazione. Ha anche illustrato il processo di gestione dei cambiamenti di scope, e il documento da utilizzare per la richiesta, la valutazione, l'approvazione e la comunicazione dei cambiamenti, ovvero il project impact statement (PIS).
    \item \textbf{Discussione e chiarimenti}: il project manager ha aperto una discussione tra i partecipanti, per rispondere alle domande, risolvere i dubbi, e concordare le priorità del progetto. Ha anche raccolto i feedback e le aspettative del cliente e del team di sviluppo, e ha cercato di creare un clima di fiducia e impegno reciproco.
\end{itemize}

\section{Regole Operative per il Team}

\section{Gestione delle Comunicazioni}

\section{Gestione dei Cambiamenti di Scope}
