\chapter{Launching}
La fase di avvio del progetto è quella in cui si recluta il team di sviluppo, si conduce il kick-off meeting, si definiscono le regole operative per il team, si gestiscono le comunicazioni e i cambiamenti di scope. In questa fase, il project manager ha il compito di motivare, coordinare e guidare il team verso il raggiungimento degli obiettivi del progetto.

\section{Scelta del Team di Sviluppo}
Per il progetto in questione, il team di sviluppo è stato scelto tenendo conto delle competenze necessarie per realizzare il sistema CityTwin. Il team è composto da:

\begin{itemize}
    \item Un architetto software, che ha il ruolo di progettare l’architettura del sistema, definire le specifiche tecniche, coordinare e supportare gli sviluppatori, garantire la qualità e la sicurezza del codice.
    \item Un esperto in sicurezza informatica, che ha il ruolo di valutare e mitigare i rischi di attacchi informatici, implementare le misure di protezione dei dati e delle comunicazioni, testare la robustezza e la resilienza del sistema.
    \item Due esperti in IoT e sensoristica, che hanno il ruolo di selezionare e configurare i sensori, integrare i dispositivi con il sistema, gestire i dati raccolti, ottimizzare le prestazioni e il consumo energetico.
    \item Otto sviluppatori generici, che hanno il ruolo di sviluppare le funzionalità del sistema, seguendo le indicazioni dell'architetto software e dell'esperto in sicurezza informatica, utilizzando i linguaggi e gli strumenti più adatti.
\end{itemize}

Il team di sviluppo è già consolidato, in quanto i suoi membri hanno lavorato insieme in precedenti progetti, e hanno quindi una buona conoscenza reciproca, una forte coesione e una elevata produttività.

\section{Kick-Off Meeting}

\section{Regole Operative per il Team}

\section{Gestione delle Comunicazioni}

\section{Gestione dei Cambiamenti di Scope}
